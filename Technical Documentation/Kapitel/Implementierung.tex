\section{Implementierung}

    \subsection{Projektaufbau}
React Native bietet eine Option einfach ein neues Projekt anzulegen. Der Befehl \textit{react-native init "projectname"} geniert alle benötigten Dateien und Ordner um ein Projekt zu starten. Dies ermöglicht einen schnellen, einfachen Einstieg in React Native. Die initiale Ordnerstruktur wurde im Zuge der React Native Arbeit näher erläutert. Für größere Projekte ist diese Struktur allerdings nicht ideal. Die initial generierten Dateien \textit{index.android.js} und \textit{index.ios.js} dienen den nativen Android und iOS Apps als Ausgangspunkt. Mit dem Befehl \textit{AppRegistry} kann die App dort registriert werden. Die Dateien enthalten zu Anfang allerdings viel doppelten Code. Um die Codequalität zu verbessern und zu vereinheitlichen wurde deshalb die Ordnerstruktur für das Projekt angepasst. Ziele für die Umstrukturierung war die maximale Wiederverwendbarkeit von Code. Die Graphik \ref{lst:directory_structure} zeigt die verwendete Ordnerstruktur.

  \lstdefinestyle{tree}{
      literate=
      {├}{{\smash{\raisebox{-1ex}{\rule{1pt}{\baselineskip}}}\raisebox{0.5ex}{\rule{1ex}{1pt}}}}1 
      {─}{{\raisebox{0.5ex}{\rule{1.5ex}{1pt}}}}1 
      {└}{{\smash{\raisebox{0.5ex}{\rule{1pt}{\dimexpr\baselineskip-1.5ex}}}\raisebox{0.5ex}{\rule{1ex}{1pt}}}}1 
    }
    
    \begin{lstlisting}[style=tree]
    .
    ├── .babelrc
    ├── .buckconfig
    ├── .eslintrc.json
    ├── .flowconfig
    ├── .watchmanconfig
    └── __tests__/
    ├── android/
    ├── app/
    ├── index.android.js
    ├── index.ios.js
    ├── ios/
    ├── node_modules/
    └── package.json
    \end{lstlisting}
    \vspace{-0.5 cm}
    \begin{listing}[H]
        \caption{Initiale Verzeichnisstruktur eines React Native Projekts}
        \label{lst:directory_structure}
    \end{listing}
    
 Die Android und iOS Ordner enthalten den Nativen Code für die Apps. Die React Native Entwicklung der App befindet sich fast ausschließlich in dem Ordner \textit{app}. Die Entwicklung von Tests befindet sich in \textit{\_\_tests\_\_}. Die verwendeten Bibliotheken stehen in der \textit{package.json} Datei und deren Versionen können dort angepasst werden. Die Konfiguration des JavaScript-Compilers befindet sich in der \textit{.babelrc} Datei. \\
 
 In der Graphik \ref{lst:app_directory_structure} sind die Unterordner des Ordners app zu sehen. Die \textit{index.js} Datei wird von den beiden \textit{index.android.js} und \textit{index.ios.js} Dateien aufgerufen und exportiert dorthin.  Der erste zu erwähnende Ordner ist \textit{components}. Dieser dient der Strukturierung von wiederverwendbaren Komponenten, wie beispielsweise Buttons. In diesem Ordner liegt auch eine index.js Datei. Diese verwaltet was exportiert werden soll aus dem Ordner. \\
 
 In dem Ordner \textit{config} befindet sich wiederverwendbare Styles, die so von überall her genutzt werden können. Zusätzlich befinden sich in der \textit{images.js} Datei Pfade für die verwendeten Bilder. Damit können diese zentral verwaltet werden. Auch in diesem Ordner liegt eine index Datei zur Export-Verwaltung. Die Bilder selbst werden im Ordner \textit{images} gespeichert.\\
 
 %TODO
 Database, models, redux
 
 Die Routen reflektieren die verschiedenen Views in der App. Die Views werden in \ref{views} näher erklärt. In dem Ordner \textit{routes} werden die Views implementiert. Hier befindet sich der Hauptteil der Implementierung der App. Die Datei \textit{router.js} ist zuständig zwischen den verschiedenen Views zu navigieren und den aktuellen View anzuzeigen. Die Datei wird von der index.js Datei importiert. Dies sorgt für eine Übersichtliche Entwicklung aller Teile der App.
 
    \begin{lstlisting}[style=tree]
    .
    ├── components
    ├── config
    ├── database
    ├── images
    ├── index.js
    ├── router.js
    ├── models
    ├── network
    ├── redux
    └── routes

    \end{lstlisting}
    \vspace{-0.5 cm}
    \begin{listing}[H]
        \caption{Initiale Verzeichnisstruktur eines React Native Projekts}
        \label{lst:app_directory_structure}
    \end{listing}
        
        
        
    \subsection{Mobile Applikation}
        
        \subsubsection{Installation}
Um das Projekt mit React native ausführen zu können, muss als Erstes Node installiert sein. Zusätzlich sollte Watchman installiert werden. Dies beobachtet Dateiänderungen und sorgt für eine bessere Performance \cite{facebook_inc._start_2017}.  Als nächstes kann mit dem Package Manager npm das Paket \textit{react-native-cli},welches das  React Native command line interface ist installiert werden. \\

Um eine React Native iOS App zu starten muss nur noch im entsprechenden Verzeichnis \textit{react-native run-ios} ausgeführt werden. Alternativ kann die App auch über das Xcode-Projekt im ios Ordner der App gestartet werden. Dabei kann eine iOS App nur auf dem MacOS-System gestartet werden. \cite{facebook_inc._start_2017}\\

Zum Ausführen einer Android App kann Android Studio oder eine ähnliche Umgebung installiert werden. Android Studio enthält das Android SDK und AVD, den Emulator. Diese werden zum ausführen der App benötigt \cite{facebook_inc._start_2017}. Im Ordner der React Native App kann nun die App mit dem Befehl \textit{react-native run-android} gestartet werden. \\

Vor dem Starten der App sollte der Befehl \textit{npm install} ausgeführt werden, um alle Abhängigkeiten zu installieren. Zusätzlich muss der Image-Crop-Picker noch installiert werden. Dieser dient zum auswählen der Bilder in der App. Mit npm kann der Befehl \textit{npm i react-native-image-crop-picker --save} ausgeführt werden. Anschließend wird noch das Linking durchgeführt mit dem Befehl \textit{react-native link react-native-image-crop-picker} \cite{pusic_crop_2017}. In Xcode muss dann noch unter \textit{Deployment Info} das \textit{Deployment Target} auf 8.0 gesetzt werden. Unter dem Punkt \textit{Embedded Binaries} werden die beiden Dateien \textit{RSKImageCropper.framework} und \textit{QBImagePicker.framework} noch hinzugefügt \cite{pusic_crop_2017}. 

        \subsubsection{Verwendete Bibliotheken}
Um eine einfache und übersichtliche  Navigation zu gewährleisten würde die Bibliothek \textit{ExNavigation} verwendet. Die Bibliothek wird im Moment in die reactjs Organisation eingebunden und in der nächsten Version unter dem Namen \textit{react-navigation} verfügbar sein \cite{Vatne_exnavigation_2017}. Die Bibliothek ermöglicht eine einfachere Navigation als die Standard React Native Navigation im Moment. Mit wenig Entwicklungsaufwand konnte so die Navigation zwischen den Views ermöglicht werden. \\

%TODO
react-redux, realm, fetch-blob, ESLint, image-crop-picker
\subsection{Server}
    
    \subsubsection{Installation / Ausführung}
    \subsubsection{Security}

\subsection{Microcontroller}
    
    