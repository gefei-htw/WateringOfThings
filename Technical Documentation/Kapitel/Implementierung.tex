\section{Implementierung}

    \subsection{Projektaufbau}
Ordnerstruktur

  \lstdefinestyle{tree}{
      literate=
      {├}{{\smash{\raisebox{-1ex}{\rule{1pt}{\baselineskip}}}\raisebox{0.5ex}{\rule{1ex}{1pt}}}}1 
      {─}{{\raisebox{0.5ex}{\rule{1.5ex}{1pt}}}}1 
      {└}{{\smash{\raisebox{0.5ex}{\rule{1pt}{\dimexpr\baselineskip-1.5ex}}}\raisebox{0.5ex}{\rule{1ex}{1pt}}}}1 
    }
    
    \begin{lstlisting}[style=tree]
    .
    ├── .babelrc
    ├── .buckconfig
    ├── .eslintrc.json
    ├── .flowconfig
    ├── .watchmanconfig
    └── __tests__/
    ├── android/
    ├── app/
    ├── index.android.js
    ├── index.ios.js
    ├── ios/
    ├── node_modules/
    └── package.json
    \end{lstlisting}
    \vspace{-0.5 cm}
    \begin{listing}[H]
        \caption{Initiale Verzeichnisstruktur eines React Native Projekts}
        \label{lst:idirectory_structure}
    \end{listing}
    
App Unterordner
    \begin{lstlisting}[style=tree]
    .
    ├── components
    ├── config
    ├── database
    ├── images
    ├── index.js
    ├── lib/
    ├── models/
    ├── network
    └── routes

    \end{lstlisting}
    \vspace{-0.5 cm}
    \begin{listing}[H]
        \caption{Initiale Verzeichnisstruktur eines React Native Projekts}
        \label{lst:idirectory_structure}
    \end{listing}
        
        
        
    \subsection{Mobile Applikation}
        
        \subsubsection{Installation}
        \subsubsection{Verwendete Bibliotheken}

\subsection{Server}
    
    \subsubsection{Installation / Ausführung}
    \subsubsection{Security}

\subsection{Microcontroller}
    
    