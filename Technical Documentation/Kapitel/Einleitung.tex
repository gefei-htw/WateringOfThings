\section{Einleitung}
Pflanzen perfekt zu wässern ist ein Kunst für sich. Jede Pflanze braucht eine andere, für sie optimale Menge an Wasser. Ist man für eine längere Zeit nicht zu Hause, können Pflanzen mit hohem Wasserbedarf leicht vertrocknen. Um auch im Urlaub seine Pflanzen gut versorgt zu wissen, ist eine Bewässerungsapplikation eine gute Option. Mithilfe dieser Bewässerungsapp soll es möglich sein, Pflanzen auch ferngesteuert gießen zu können. Da Pflanzen eine individuelle Mengen an Wasser brauchen, sollte es mit der App möglich sein jede Pflanze individuell bewässern zu können. Ziel der App ist eine optimale Versorgung von Zimmerpflanzen. Die benötigten Hardwarekomponenten können günstig erworben werden. Da die Pflanzen mithilfe der App nach einem Urlaub noch gesund sind, entsteht so schnell ein Kostenvorteil für den Nutzer, da Pflanzen nicht neu gekauft werden müssen. Viele Menschen vereisen für immer längere Zeit. Seine Pflanzen während dieser Zeit gut versorgt zu wissen ist das Ziel der \textit{WateringOfThings}-App.