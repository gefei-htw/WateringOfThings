\section{Projektaufbau}

\subsection{Entwicklungsumgebung}
React Native lässt sich in jedem Texteditor entwickeln. Es gibt aber auch spezielle Entwicklungsumgebungen wie \textit{Nuclide}, welches ein Package für den Texteditor Atom ist. Allerdings fiel beim Text auf, dass dies Atom sehr langsam machte und für die Entwicklung kaum Vorteile brachte. Deco ist eine weitere mögliche Entwicklungsumgebung. Diese beinhaltet allerdings wenige Features und keine ES-Lint Integration, dies macht eine qualitative Entwicklung einer App schwierig. Die besten Erfahrungen konnten mit Atom in Kombination mit Es-Lint und weiteren React Native Packages gemacht werden. 

\subsection{Verzeichnisstruktur}

\subsection{Ausführen der Anwendung}

\subsection{Debugging}