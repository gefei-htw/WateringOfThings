\section{React Native}

\subsection{Components}
React Native und React konstruieren Applikationen mit Hilfe von \textit{Components}. React Native verwendet dabei native Komponenten von Android und iOS, während React Web Components nutzt  \cite{facebook_inc._start_2017}. Die \textit{React.Component} Klasse von React wird als abstrakte Basisklasse verwendet. Alle eigenen Klassen werden im Normalfall von dieser Klasse abgeleitet. 

\subsection{Props}

\subsection{State}

\subsection{Styling}
% hier mehr auf farben etc, nicht so sehr auf Layouting -> siehe unten

\subsection{Layout}
% Flexbox

\subsection{Views}
% ListView ScrollView

\subsection{Plattform-Spezialisierung}
https://facebook.github.io/react-native/docs/platform-specific-code.html

\subsection{Integration nativer Module}
%Ist das nicht ein unterpunkt von Plattform Sepezialisierung?
% Linking? 

\subsection{Networking}
%Oder wo anders, aber eigentlich hier?
