\section{Einleitung}
React Native entstand als ein Projekt in einem Facebook Hackathon im Jahr 2013. Daraufhin wurde das Projekt bei Facebook weiterentwickelt und erste kleine Testapps entwickelt. Im Juli 2014 begann die Entwicklung der ersten lauffähigen App mit React Native, eine Ads Manager App für iOS. Das Ziel der App war die gleiche User Experience zu haben wie eine App in Objective-C. Kurze Zeit später wurde das erste React Native Android Team bei Facebook zusammengestellt, um native Komponenten auch für Android zu entwickeln. Die iOS Version der Ads Manager App wurde im Februar 2015 veröffentlicht und im September konnte auch die erste Android Version des Ads Managers gelaunched werden. \\

React Native ist seit März 2015 Open Source. Dies ermöglicht eine schnelle Weiterentwicklung auch außerhalb von Facebook und eine große Vielfalt an Bibliotheken und Features. Um Neuerungen in React Native so früh wie möglich zur Verfügung zu stellen wird im Moment alle zwei Wochen ein neues Release veröffentlicht \cite{Konicek_review_2016}. Mehr als 9000 Commits von über 1000 Menschen sind bislang in React Native eingegangen und nach nur einem Jahr als Open Source Projekt ist eine große Community entstanden, die React Native rasant weiterentwickelt und verbessert.