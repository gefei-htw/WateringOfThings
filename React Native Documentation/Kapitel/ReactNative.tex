\section{React Native}

\subsection{Components} \label{component}
React Native und React konstruieren Applikationen mit Hilfe von \textit{Components}. React Native verwendet dabei native Komponenten von Android und iOS, während React Web Components nutzt  \cite{facebook_inc._getting_2017}. Für die Implementierung eigener Komponenten wird die \textit{React.Component}-Klasse von React als abstrakte Basisklasse verwendet. Mithilfe von Components kann das User Interface der Anwendung in unabhängige, wiederverwendbare Teile gegliedert werden. Die Components sind mit JavaScript-Funktionen vergleichbar, welche Daten als Eingabe erhalten und deren Visualisierung zurück geben \cite{facebook_inc._components_2017}. Im Beispiel \ref{lst:component} wird die Komponente \textit{HomeView} angelegt. Bei der Definition von Komponenten ist die Implementierung der \textit{render}-Funktion Pflicht. \textit{AppRegistry.registerComponent} definiert den Anfangspunkt der App für React Native und bekommt die hierarchisch höchste Komponente übergeben \cite{facebook_inc._getting_2017}. 

\subsection{Props}
Um Kindkomponenten dynamisch anzupassen, können Elternkomponenten Attribute innerhalb eines \textit{props}-Objektes übergeben \cite{facebook_inc._components_2017}. Props ermöglichen es, dass eine Component an verschiedenen Stellen mit verschiedenen Daten wiederverwendet werden kann \cite{facebook_inc._props_2017}. Es ist allerdings nur lesender Zugriff auf Props möglich, d.h. sie können nicht modifiziert werden. Im Beispiel \ref{lst:props} wird der Name der Pflanze übergeben, um in der Component \textit{PlantView} verschiedene Pflanzennamen anzeigen zu können. Dadurch kann die Component leicht wiederverwendet werden. Es können sowohl eigene Components als auch Components von React Native, wie beispielsweise \textit{<Text>}, verwendet werden. Ändern sich übergebene Properties werden betroffene Teile der View automatisch neu gerendert.
%Dies ermöglicht eine besondere Freiheit beim Gestalten der App. Sollte eine vorgegebene Component nicht passen, kann man leicht eine neue kreieren und verwenden, wie in diesem Fall PlantView. Die \textit{<View>} Component ist das Elternelement der geschachtelten Components PlantView. 

\begin{listing}[H]
    \begin{minted}{js}
import React, { Component } from 'react';
import { AppRegistry, Text } from 'react-native';

class HomeView extends Component {
    render() {
        return (
            <Text>Hello</Text>
        );
    }
}
AppRegistry.registerComponent('HomeView', () => HomeView);
    \end{minted}
    \caption{Erstellung einer eigenen Komponente, die den Text \textit{Hello} ausgibt}
    \label{lst:component}
\end{listing}
\begin{listing}[H]
    \begin{minted}{js}
import React, { Component } from 'react';
import { Text, View } from 'react-native';

class PlantView extends Component {
    render() {
        return (
            <Text>My plant: {this.props.name}</Text>
        );
    }
}

class PlantsView extends Component {
    render() {
        return (
            <View>
                <PlantView name='Basilikum' />
                <PlantView name='Kaktus' />
            </View>
        );
    }
}

    \end{minted}
    \caption{Definition einer \textit{PlantView}-Component, die den anzuzeigenden Pflanzenname als Property übergeben bekommt}
    \label{lst:props}
\end{listing}

\subsection{State}
\textit{State} ermöglicht es die Daten einer Komponente zu ändern und damit ein erneutes Rendern der Ansicht auszulösen. State sollte im Normalfall im Konstruktor initialisiert werden und mit \textit{setState} können die Daten dann später bearbeitet werden. Im Beispiel \ref{lst:state} wird ein blinkender Text erstellt. Dies zeigt auch anschaulich den Unterschied zwischen Props und State. Ein Property ist hier der Text, der ausschließlich von der Elternkomponente geändert werden kann. Die Kindkomponente kann jedoch mittels des States die eigene Anzeige bestimmen. In Beispiel \cite{facebook_inc._state_2017}. Ein praktisches Beispiel für State ist der Zugriff auf eine REST API. Beispielsweise kann im State ein Feld \textit{isLoadingData} definiert werden, anhand dessen das Rendern einer Warteanimation gesteuert wird.

\begin{listing}[H]
    \begin{minted}{js}
import React, { Component } from 'react';
import {Text, View } from 'react-native';

class Blink extends Component {
    constructor(props) {
        super(props);
        this.state = { showText: true };
        
        // Toggle the state every second
        setInterval(() => {
            this.setState({ showText: !this.state.showText });
        }, 1000);
    }

    render() {
        return <Text>{this.state.showText && this.props.text}</Text>
    }
}

class BlinkApp extends Component {
    render() {
        return (
            <View>
                <Blink text='blink' />
            </View>
        );
    }
}

    \end{minted}
    \caption{Definition einer View, die einen blinkenden Text anzeigt \cite{facebook_inc._state_2017}}
    \label{lst:state}
\end{listing}


\subsection{Styling}
Das Definieren von Styles ist an CSS angelehnt, geschieht aber ebenfalls in JavaScript. Dabei können die meisten der in CSS bekannten Styles wiederverwendet werden. Alle von der React.Component-Klasse abgeleiteten Elemente akzeptieren den Prop \textit{Style} \cite{facebook_inc._style_2017} mit dessen Hilfe die Darstellung der Komponente angepasst werden kann. Um die Styles zu bündeln und übersichtlich zu halten, empfiehlt es sich, ein Stylesheet mit \textit{StyleSheet.create} zu erstellen \cite{facebook_inc._style_2017}. 

\begin{listing}[H]
    \begin{minted}{js}
import React, { Component } from 'react';
import {StyleSheet, Text, View } from 'react-native';

class HomeView extends Component {
    render() {
        return (
            <View style={styles.container}>
                <Text style={styles.textStyle}> gestylter Text </Text>
            </View>
        );
    }
}

const styles = StyleSheet.create({
    container: {
        backgroundColor: 'grey',
    },
    textStyle: {
        color: 'black',
        fontSize: 30,
        textAlign: 'center',
    }
});    
    \end{minted}
    \caption{Customizen von Komponenten mit Hilfe der \textit{style}-Property}
    \label{lst:style}
\end{listing}

\subsection{Layout}
Smartphones unterscheiden sich in ihren Abmessungen und Bildschirmgrößen. React Native bietet daher die Möglichkeit ein dynamisches Layout mittels dem Style \textit{flex} zu erstellen \cite{facebook_inc._flex_2017}.  Damit werden Abmessung nicht absolut definiert, sondern anhand von Gewichten. Wird beispielsweise jeder Component \textit{flex: 1} als Style übergeben, teilen sich die Komponenten den Platz gleichmäßig auf. Je größer die vergebene flex-Zahl ist, desto mehr Platz erhält diese Komponente im Verhältnis zu anderen Components innerhalb des gemeinsamen Parent-Element \cite{facebook_inc._flex_2017}. 

\begin{listing}[H]
    \begin{minted}{js}
import React, { Component } from 'react';
import {View } from 'react-native';

class HomeView extends Component {
    render() {
        return (
            <View style={{flex: 1}}>
                <View style={{flex: 1}} />
                <View style={{flex: 2}} />
            </View>
        );
    }
};

    \end{minted}
    \caption{Einsatz von Flexbox für eine dynamische Größenanpassung der Komponenten}
    \label{lst:flex}
\end{listing}
\hyphenation{Platt-form}
Damit kann ein einheitliches Layout auf unterschiedlichen Geräten gewährleistet werden. Weitere Styles zum Verfeinern des Layouts sind \textit{flexDirection}, \textit{alignItems} und \textit{justifyContent}. Mit FlexDirection kann die Ausrichtung der Objekte bestimmt werden, d.h. die Objekte können vertikal (\textit{column}) oder horizontal (\textit{row}) angeordnet werden. Der Standardwert für Flexboxen ist horizontal. Die Eigenschaft \textit{justifyContent} regelt dabei die Verteilung der Komponenten in Flex-Richtung, d.h. beispielsweise dass sich die Komponenten in der Mitte oder am Rand konzentrieren, während \textit{alignItems} die Ausrichtung der Kind-Komponenten in der jeweils anderen Achse regelt \cite{facebook_inc._flex_2017}. \\

%Die Verteilung der Components wird mit justifyContent geregelt. Es ist möglich die Objekte links-, rechtsbündig oder mittig anzuordnen oder mit Abstand um das Objekt oder zwischen den Objekten \cite{facebook_inc._flex_2017}.

\subsection{Plattform-Spezialisierung}
Auch wenn es das Ziel ist, möglichst viel Code zwischen den Plattformen wiederzuverwenden, gibt es Situationen in denen zwischen den Plattformen differenziert werden muss. Dies kann beispielsweise nötig werden, falls die App an das Look-And-Feel der jeweiligen Plattformen angepasst werden solll, um damit die User-Experience zu erhöhen.
React Native bietet für die Plattformunterscheidung zwei Möglichkeiten an. Zum einen kann die Plattform abgefragt werden, zum anderen kann Code in Dateien mit Plattform-spezifischen Endungen abgelegt werden. Beide Möglichkeiten werden nachfolgend gezeigt.

\begin{listing}[H]
    \begin{minted}{js}
import { Platform, StyleSheet } from 'react-native';

const styles = StyleSheet.create({
    // Das Element hat in iOS einen weißen Hintergrund und einen blauen in Android
    backgroundColor: (Platform.OS === 'ios') ? 'white' : 'blue',
});
    \end{minted}
    \caption{Abfragen der Plattform}
    \label{lst:platform_os}
\end{listing}

\begin{listing}[H]
    \begin{minted}{js}
// Datei: colors.ios.js
export default defaultBackgroundColor = 'white';

// Datei: colors.android.js
export default defaultBackgroundColor = 'blue';
 
// Datei: style.js
// Der Import erfolgt ohne Plattform-spezifische Dateiendung
import defaultBackgroundColor from './colors';
 
const styles = StyleSheet.create({
    // Das Element hat in iOS einen weißen Hintergrund und einen blauen in Android
    backgroundColor: defaultBackgroundColor
});
     
    \end{minted}
    \caption{Automatische Auswahl der Datei anhand der Dateiendung}
    \label{lst:platform_file}
\end{listing}

\subsection{Networking}
Zum Ansprechen einer REST API stellt React Native die \textit{Fetch-API} bereit. \cite{facebook_inc._fetch_2017}. Um Daten mittels eines GET-Requests von einer URL zu laden, wird nach folgendem Schema vorgegangen:
\begin{listing}[H]
    \begin{minted}{js}
fetch('https://mywebsite.com/endpoint/', {
    method: 'GET',
    headers: {
        'Accept': 'application/json',
    },
})
.then((responseJSON) => responseJSON.json())
.then((decoded) => ...)
.catch((eventualException) => ...);
    \end{minted}
    \caption{GET-Request mit der Fetch-API \cite{facebook_inc._fetch_2017}}
    \label{lst:get_fetch}
\end{listing}

Beim Senden von Daten an die API müssen diese in das passende JSON-Format umgewandelt sowie das Datenformat im Header spezifiziert werden:

\begin{listing}[H]
    \begin{minted}{js}
fetch('https://mywebsite.com/endpoint/', {
    method: 'POST',
    headers: {
        'Accept': 'application/json',
        'Content-Type': 'application/json',
    },
    body: JSON.stringify({
        firstParam: 'yourValue',
        secondParam: 'yourOtherValue',
    })
})
.then((responseJSON) => ... )
.catch((eventualException) => ...);
    \end{minted}
    \caption{POST-Request mit der Fetch-API \cite{facebook_inc._fetch_2017}}
    \label{lst:fetch}
\end{listing}

