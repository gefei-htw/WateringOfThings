\section{Résumé}
React Native ermöglicht es einfach und rasch Apps für die Plattformen iOS und Android zu entwickeln. Dabei sind lediglich JavaScript- und React-Kenntnisse notwendig, sodass das Paradigma \textit{Learn once, write anywhere} gilt. Dank der JSX-Syntax ist weiterhin eine intuitive UI-Gestaltung mittels JavaScripts möglich. Auch wenn React Native sich nur auf einen geringen Funktionsumfang konzentriert, ist die Erweiterung um zusätzliche Features dank der sehr engagierten Community im Regelfall möglich. Hierbei ist anzumerken, dass insbesondere zu Beginn die Recherche und Auswahl der geeigneten Bibliotheken die Produktivität hemmt. Aktuell werden jedoch schon mehrere Bibliotheken entwickelt, welche viele grundlegende Elemente zu Verfügung stellen, die auch schon plattform-spezifisches Styling aufweisen. Auch sind React Native eigene Komponenten sowie externe Bibliotheken nicht immer vollständig dokumentiert, sodass es teilweise nötig ist den Quellcode zum Verständnis heranzuziehen. Da React Native sehr jung ist, ist die Entwicklungsgeschwindigkeit und die damit verbundene Anzahl an Änderungen sehr hoch. React Native-Entwickler sollten sich daher darauf einstellen, dass Updates der Versionen zu einer funktionsunfähigen Applikation führen. 
React Native ist schon jetzt eine sehr gute Alternative um native Apps zu schreiben, ohne dabei auf Codewiederverwendung zu verzichten.